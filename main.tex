\documentclass[12pt]{article}

  \usepackage[english]{babel}
  \usepackage{hyperref}
  \usepackage{fancyhdr}
  \usepackage[dvipsnames]{xcolor}
  \usepackage{listings}
  \usepackage{parcolumns}
  \usepackage{algorithm}
  \usepackage{algorithmicx}
  \usepackage{algpseudocode}
  \usepackage{enumitem}
  \usepackage{geometry}
  \usepackage{graphicx}
  \usepackage{enumitem}
  \usepackage{csquotes}
  \usepackage{bookmark}
  \usepackage{mdframed}
  \usepackage{mathtools}
  \usepackage{amsmath}
  \usepackage{amsthm}
  \usepackage[toc]{appendix}
  \usepackage[
    backend=biber,%
    style=ieee%
  ]{biblatex}

  \addbibresource{main.bib}
  
  \theoremstyle{definition}
  \newtheorem*{defn*}{Definition}
  \theoremstyle{plain}
  \newtheorem*{equ*}{Equation}

  \algdef{SE}[VARIABLES]{GVariables}{EndGVariables}
    {\algorithmicvariables}
    {\algorithmicend\ \algorithmicvariables}
  \algnewcommand{\algorithmicvariables}{\textbf{global variables}}
  
  \algdef{SE}[VARIABLES]{LVariables}{EndLVariables}
    {\algorithmiclvariables}
    {\algorithmicend\ \algorithmiclvariables}
  \algnewcommand{\algorithmiclvariables}{\textit{local variables}}

  \renewcommand{\algorithmicrequire}{\textbf{Input:}}
  \renewcommand{\algorithmicensure}{\textbf{Output:}}
  \renewcommand\thealgorithm{}

  \makeatletter
  \def\mathcolor#1#{\@mathcolor{#1}}
  \def\@mathcolor#1#2#3{%
    \protect\leavevmode
    \begingroup
      \color#1{#2}#3%
    \endgroup
  }
  \makeatother


  % Image Directory
  \graphicspath{ {screenshots/} }
  % Hyperlink Setup
  \hypersetup{
    colorlinks = true,
    urlcolor = blue,
    linkcolor = blue,
    citecolor = blue
  }
  % Syntax-Highlight for Code Snippets
  \lstset{
    backgroundcolor=\color{white},
    breaklines=true,
    captionpos=b,
    frame=tb,
    tabsize=4,
    % numbers=left,
    showstringspaces=false,
    commentstyle=\color{Red},
    keywordstyle=\color{Aquamarine},
    stringstyle=\color{ForestGreen}
  }

  % Page and Text Layout
  \pagestyle{fancy}
  \geometry{%
  a4paper,%
  top=2.5cm,%
  bottom=2.5cm,%
  left=2.5cm,%
  right=2.5cm%
  }

  \newenvironment{ldefinitions}
    {\left.\begin{aligned}}
    {\end{aligned}\right\rbrace}

  \title{Module 3 Concept Discussion}
  \author{Ashton Hellwig}
  \date{\today}
  \rhead{CSC160 Concept Discussion}

\begin{document}
  \maketitle
  \tableofcontents
  \lstlistoflistings
  \newpage


  \part{Initial Post}

    \section{Research Prompt}
      \begin{quote}
        Look forward into Chapter 15 for Recursive
        Functions.
        \begin{itemize}
          \item What is a recursive function and what are 2 applications of
            recursive functions you can think of?
        \end{itemize}
      \end{quote}

    \section{Response}
      \subsection{What is a Recursive Function?}
        \textbf{Recursion} is the process of solving a problem by reducing it to
          smaller versions of itself. There are two types of recursion -- direct
          and indirect. For a function to be considered
          \textbf{directly recursive}, it must \textit{call itself}. Much like
          a recursive function, a \textbf{recursive definition} is a definition
          in which something is defined in terms of smaller versions of itself,
          for example:
          \begin{equation}
            \begin{ldefinitions}
              & \text{\textbf{Base Case}} & 0! &= 1 \\
              &\text{\textbf{General Case}} & n!&=n\times (n-1)!\text{ if }n>0
              \hspace{20pt}
            \end{ldefinitions}
            \text{\textbf{Recursive Definition}}
          \end{equation}

        From the textbook \autocite{malik_2015}:
        \begin{quote}
          In this definition, $0!$ is defined to be $1$, and if $n$ is greater
            than $0$, first we find $(n-1)!$ and then multiply it by $n$. To
            find $(n-1)!$, we apply the definition again. If $(n-1)>0$ then we
            use equation 2, otherwise, we use equation 1.

          \begin{mdframed}[backgroundcolor=yellow!20]
            \textbf{Example}: Find $3!$
            \begin{align*}
              \mathbf{n} &= 3 \\
              3!         &= 3\times 2 &\text{Since n is 3, use equation 2} \\
              2!         &= 2\times 1 &\text{Since n is 2, use equation 2} \\
              1!         &= 1\times 1 &\text{Since n is 1, use equation 2} \\
              \shortintertext{Finally, use equation 1 since after the previous step}
              \mathbf{n} &= 0 \\
              0! &= 1 \\
              1! &= 1 & \text{Substituting $0!$ into $1!$ gives us this} \\
              2! &= 2\times 1!=2\times 1=2 & \text{Giving us this} \\
              3! &= 3\times 2!=3\times 2=6 & \text{Giving us the final answer}
            \end{align*}
          \end{mdframed}
        \end{quote}

        Recursive functions are used to implement recursive algorithms. Below is
          an example of how the equation above is implemented in C++
          \autocite{malik_2015}. Note how the function below is a
          \textit{directly} recursive function 

          \begin{lstlisting}[language=c++,caption=Factorial Function,numbers=left]
// begin recursive definition
int fact(int number) {
  if (number == 0) { // base case
    return 1;
  } else {
    return number * fact(number - 1); // general case
  }
} // end recursive definition
          \end{lstlisting}

          \begin{algorithm}[H]
            \begin{algorithmic}[1]
              \Require{$n$, an $integer$} \Comment{$n$ is any integer}
              \Ensure{$!n$, \text{the factorial of n}} \Comment{Return value}
              \Statex
              \GVariables
                  \State $n\gets integer$
              \EndGVariables
              \Statex
              \Procedure{factorial}{$n$} \Comment{Recursive definition}
                \If{$n==0$} \Comment{Base Case}
                  \State \Return $1$
                \ElsIf{$n>0$} \Comment{General Case}
                  \State \Return $n\times \text{\Call{factorial}{$n-1$}}$
                \EndIf
              \EndProcedure
            \end{algorithmic}
            \caption{Factorial Function Algorithm}
          \end{algorithm}

        Designing a recursive function requires knowledge of the constraints
          of the problem along with identification of the base and general
          cases. It is important that care is taken to ensure that the
          function`s recursive calls ultimately lead to a base case. This will
          avoid \textbf{infinite recursion}, which occurs when a recursive call
          results in another recursive call, leading to the perpetual execution
          of the function.

      \subsection{Recursion Applications}
        It is best to use recursive functions only when the definition of the
          problem is inherently recursive, due to the overhead that is brought
          with recursive functions due to each call requiring new memory
          allocation and will not deallocate until the recursion is complete. If
          a problem seems that it could be easily solved with a
          selection/control structure, that would be a better way of solving it.
        \subsubsection{Sorting Algorithms}
          Recursion is incredibly useful for problems which require a sorting
            algorithm. Most larger technical companies (Facebook, Apple,
            Microsoft, Amazon, Google, Netflix) have prospective employees
            perform a white-board coding session in which they are given a
            problem and told to write an algorithm to solve it. I will use the
            bubble sort algorithm in recursive manner to illustrate the
            idea. Bubble sort compares two elements at a time, swapping them in
            the event that the one on the lower index is greater than the value
            it is compared with. It is meant for lists and arrays.
          
          \begin{algorithm}[H]
            \caption{Recursive bubble sort}
            \begin{algorithmic}[1]
              \Procedure{swapElements}{$firstElement$, $secondElement$}
                \State $tempValue\gets firstElement$
                \State $firstElement\gets secondElement$
                \State $secondElement\gets tempValue$
              \EndProcedure
              \State
              \Procedure{bubbleSort}{$array$, $number$} \Comment{$n=$ size of array}
                \State $i\gets 0$
                \State
                \If{number == 1} \Comment{Base case}
                    \State \Return
                \EndIf
                \State
                \For{$i<number-1$; $i=i+1$}
                  \If{$array[i]<array[i+1]$}
                    \State \Call{swapElements}{$array[i]$, $array[i+1]$}
                  \EndIf
                \EndFor
                \State
                \State \Call{bubbleSort}{$array$, $n-1$}
                  \Comment{Those elements sorted, continue over each element}
              \EndProcedure
            \end{algorithmic}
          \end{algorithm}
          
        \subsubsection{Tower of Hanoi}
          The book illustrates another example problem, called the ``Tower of
            Hanoi'' \autocite{malik_2015}. This problem illustrates recursion
            by requiring the movement of disks on 3 needles. The disks are all
            stacked on top of one-another on the first needle. All of the
            disks are on the first needle in ascending circumference, so that
            the largest disk is on the bottom with the smallest on the top. Only
            one disk may move at a time, the removed disk must be placed back
            on the needles, and a larger disk cannot be above a smaller one. The
            task is to write a program which outputs the moves required to move
            all disks from the first to the third needle. In order to work with
            \textit{three} disks, we begin by moving the top $n-1$ disks from
            needle 1 to needle 2, using needle 3 to temporarily hold the smaller
            disk. Next, we move disk $n$ from needle 2 to needle 3, using
            needle 1 as we did needle 3 prior. Finally repeat the first step,
            but starting from needle 2 rather than 1. The book provides the
            following algorithm for the above statements:

            \newpage
            \begin{lstlisting}[%
              language=c++,%
              caption={Tower of Hanoi C++ Algorithm},%
              numbers=left]
void moveDisks(int count, int needle1, int needle3, int needle2) {
  if (count > 0) {
    moveDisks(count - 1, needle1, needle2, needle3);

    cout << "Move disk " << count << " from " << needle1
         << " to " << needle3 << "." << '\n';

    moveDisks(count - 1, needle2, needle3, needle1);
  }
}
            \end{lstlisting}


  \newpage
  \part{Responses}

    \section{Response 1}
      \begin{quote}
        Reply to \textbf{Tyler Olson} (\textit{Post ID: 34053893})
      \end{quote}
      What exactly do you mean when you says ``Until VS told it to stop''?
        Editors and IDEs should not really make that big of an effect on code,
        and recursive functions are supposed to stop by themselves when they
        reach the \textit{base case} of the recursive definition. You should not
        need to hit a ``stop'' button in the IDE for the program to exit
        normally. When looking at your code, from what I can see, there is no
        actual base case to the function and so you get stuck in infinite
        recursion, i.e. when a recursive function calls another recursive
        function leading to perpetual execution. Here is a recursive function
        which calculates the factorial of number:

        \begin{lstlisting}[language=c++,numbers=left]
// begin recursive definition
int fact(int number) {
  if (number == 0) { // base case
    return 1;
  } else {
    return number * fact(number - 1); // general case
  }
} // end recursive definition
        \end{lstlisting}

      As you notice in this recursive function, the base case stops the
        execution of said function when \texttt{number} is 1. Without the base
        case (as indicated by line 3), the function goes on forever.

      There are also far more examples of recursion than just factorials and
        other mathematical formulae. Tail recursion can also prevent some of
        those cases you mentioned about stack overflow and is useful in some
        instances such as duplicating a linked list (useful for interviews or
        game design). While I see how recursion can be more prone to crashing
        I feel that recursive functions are shorter and easier to read than
        their counterparts, but that may just be a preference thing.

    \section{Response 2}
      \begin{quote}
        Reply to \textbf{} (\textit{Post ID: })
      \end{quote}
      

  % Bibliography
  \newpage
  \printbibliography[
    heading=bibintoc,
    title={Bibliography}
  ]


\end{document}