\documentclass[12pt]{article}

  \usepackage[english]{babel}
  \usepackage{hyperref}
  \usepackage{fancyhdr}
  \usepackage[dvipsnames]{xcolor}
  \usepackage{listings}
  \usepackage{parcolumns}
  \usepackage{algorithm}
  \usepackage{algorithmicx}
  \usepackage{algpseudocode}
  \usepackage{enumitem}
  \usepackage{geometry}
  \usepackage{graphicx}
  \usepackage{enumitem}
  \usepackage{csquotes}
  \usepackage{bookmark}
  \usepackage{mdframed}
  \usepackage{mathtools}
  \usepackage{amsmath}
  \usepackage{amsthm}
  \usepackage[toc]{appendix}
  \usepackage[
    backend=biber,%
    style=ieee%
  ]{biblatex}

  % Bibliography Setup
  \addbibresource{main.bib}
  \newcommand{\CiteSection}[2]{%
    (\autocite{#1}, ~\S {#1})
  }

  % Theorem Environments
  \theoremstyle{definition}
  \newtheorem*{defn*}{Definition}
  \theoremstyle{plain}
  \newtheorem*{equ*}{Equation}

  % Definitions for Algorithmic Environments
  \algdef{SE}[VARIABLES]{GVariables}{EndGVariables}
    {\algorithmicvariables}
    {\algorithmicend\ \algorithmicvariables}
  \algnewcommand{\algorithmicvariables}{\textbf{global variables}}
  
  \algdef{SE}[VARIABLES]{LVariables}{EndLVariables}
    {\algorithmiclvariables}
    {\algorithmicend\ \algorithmiclvariables}
  \algnewcommand{\algorithmiclvariables}{\textit{local variables}}

  \renewcommand{\algorithmicrequire}{\textbf{Input:}}
  \renewcommand{\algorithmicensure}{\textbf{Output:}}
  \renewcommand\thealgorithm{}

  % Settings for math-mode
  \makeatletter
  \def\mathcolor#1#{\@mathcolor{#1}}
  \def\@mathcolor#1#2#3{%
    \protect\leavevmode
    \begingroup
      \color#1{#2}#3%
    \endgroup
  }
  \makeatother


  % Image Directory
  \graphicspath{ {screenshots/} }
  % Hyperlink Setup
  \hypersetup{
    colorlinks = true,
    urlcolor = blue,
    linkcolor = blue,
    citecolor = blue
  }
  % Syntax-Highlight for Code Snippets
  \lstset{
    backgroundcolor=\color{white},
    breaklines=true,
    captionpos=b,
    frame=tb,
    tabsize=4,
    % numbers=left,
    showstringspaces=false,
    commentstyle=\color{Red},
    keywordstyle=\color{Aquamarine},
    stringstyle=\color{ForestGreen}
  }

  % Page and Text Layout
  \pagestyle{fancy}
  \geometry{%
  a4paper,%
  top=1in,%
  bottom=1in,%
  left=1in,%
  right=1in%
  }

  \newenvironment{ldefinitions}
    {\left.\begin{aligned}}
    {\end{aligned}\right\rbrace}

  \title{Module 5 Concept Discussion}
  \author{Ashton Hellwig}
  \date{\today}
  \rhead{CSC160 Concept Discussion}

\begin{document}
  \maketitle
  \tableofcontents
  \lstlistoflistings
  \newpage


  \part{Initial Post}

    \section{Research Prompt}
      \begin{quote}
        Many groups use standard conventions for programming, including the
          ordering of public and private members in the class. See Examples
          10-3 through 10-5 for one such convention. What is the advantage of
          using such a standard? Do you have your own preferences?
      \end{quote}

    \section{Response}
      \subsection{Introduction}
        First, so no one will have to open the book and view Examples 10.3-10.5,
          I will include a summary (including the examples) here given in the
          \S 10.8 of the textbook \autocite{malik_2015}.
        \begin{lstlisting}[language=c++,caption={Example 10.3},label={e3}]
class clockType {
public:
  void setTime(int, int, int);
  void getTime(int&, int&, int&) const;
  void printTime() const;
  void incrementSeconds();
  void incrementMinutes();
  void incrementHours();
  bool equalTime(const clockType&) const;

private:
  int hour;
  int minute;
  int second;
}
        \end{lstlisting}
        \begin{lstlisting}[language=c++,caption={Example 10.4},label={e4}]
class clockType {
private:
  int hour;
  int minute;
  int second;

public:
  void setTime(int, int, int);
  void getTime(int&, int&, int&) const;
  void printTime() const;
  void incrementSeconds();
  void incrementMinutes();
  void incrementHours();
  bool equalTime(const clockType&) const;
}
        \end{lstlisting}
        \begin{lstlisting}[language=c++,caption={Example 10.5},label={e5}]
class clockType {
  int hour;
  int minute;
  int second;

public:
  void setTime(int, int, int);
  void getTime(int&, int&, int&) const;
  void printTime() const;
  void incrementSeconds();
  void incrementMinutes();
  void incrementHours();
  bool equalTime(const clockType&) const;
}
        \end{lstlisting}

        Right off the bat, the most notable differences are in that in Listing
          \ref{e3} the private members are placed after the public members,
          while Listing \ref{e4} places private members first and Listing
          \ref{e5} places private members first, without an access specifier.

        The book states that without an access specifier, members default to
          private and it is common practice to list \textbf{public} members
          before \textbf{private} members in order to ``focus your attention
          on the public members'' \autocite{malik_2015}.

        In order to discuss programming standards and conventions (as well as
          my preferred method of style and convention), we must go over some of
          the more common programming style-guides/conventions and contrast
          their motivations.

        \subsection{ISO Standard C++}
          Placeholder.

        \subsection{LLVM}
          Placeholder.

        \subsection{Google}
          Placeholder.

        \subsection{Mozilla}
          Placeholder.

        \subsection{Microsoft}
          Placeholder.


  \newpage
  \part{Responses}

    \section{Response 1}
      \begin{quote}
        Reply to \textbf{} (\textit{Post ID:})
      \end{quote}
      Placeholder

    \section{Response 2}
      \begin{quote}
        Reply to \textbf{} (\textit{Post ID: }) 
      \end{quote}
      Placeholder

  % Bibliography
  \newpage
  \printbibliography[
    heading=bibintoc,
    title={Bibliography}
  ]
\end{document}
